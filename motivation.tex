\section{Motivation \& Methodik}\label{sec:motivation}

\subsection{Was sind Konfliktmineralien?}

In elektronischen Geräten wie Smartphones und Laptops sind Kondensatoren im Einsatz.
Hersteller von Kondensatoren greifen dabei vor allem auf Tantal zurück. Dieses Metall wird aus einem Mineral gewonnen, welches einer der vier Konfliktminerale ist.
Förderung und Handel solcher Minerale in Konfliktgebieten können zu schweren Menschenrechtsverletzungen und Verletzungen des humanitären Völkerrechts führen. Oftmals bleibt der Reichtum nicht in den Förderländern.
~\cite{definiti26:online}

\subsection{Methodik}

Nachfolgend untersuchen wir Tantal aufgrund der häufigen Verwendung in ICT im Kontext der Nachhaltigkeit. Dabei betrachten wir Nachhaltigkeitsindikatoren der Wirtschaft, der Gesellschaft und der Umwelt in 2013 und in 2035. Dies zeigt die Nachhaltigkeit von Tantal bei stetig ansteigendem Bedarf.
Anschliessend zeichnen wir ein alternatives Szenario, bei dem die Nachfrage nach Kondensatoren aus Tantal und somit die Nachfrage durch ICT komplett verschwindet. Dies zeigt den Einfluss der Nachfrage nach Tantal durch ICT auf die Nachhaltigkeit auf.

\subsection{Zielsetzung}

Es werden die Dimensionen Wirtschaft, Gesellschaft und Umwelt werden also für folgende Situationen bewertet:

\begin{itemize}
  \item 2013: Wie war die Situation?
  \item 2035: Wie sieht die Situation bei gleichbleibend steigendem Tantalbedarf aus?
  \item 2035 (Alternativszenario): Wie sieht die Situation bei sinkendem Tantalbedarf aus?
\end{itemize}
