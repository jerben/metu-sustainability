\section{Motivation \& Methodik}\label{sec:motivation}

\subsection{Warum is Tantal interessant?}

In elektronischen Geräten wie Smartphones und Laptops sind Kondensatoren im Einsatz.
Hersteller von Kondensatoren greifen dabei vor allem auf Tantal zurück. Dieses Metall wird aus einem Mineral gewonnen, welches einer der vier Konfliktminerale ist.
Förderung und Handel dieser Minerale in Konfliktgebieten können zu schweren Menschenrechtsverletzungen und Verletzungen des humanitären Völkerrechts führen. Oftmals bleibt der Reichtum nicht in den Förderländern.~\cite{conflict_minerals}
Etwa die Hälfte des Bedarfs nach Tantal weltweit kommt aus der ICT.~\cite{why_tantal}

\subsection{Methodik}

Nachfolgend untersuchen wir Tantal aufgrund der häufigen Verwendung in ICT im Kontext der Nachhaltigkeit. Dabei betrachten wir Nachhaltigkeitsindikatoren der Wirtschaft, der Gesellschaft und der Umwelt in 2013 und in 2035. Dies zeigt die Nachhaltigkeit von Tantal bei stetig ansteigendem Bedarf.
Anschliessend zeichnen wir ein alternatives Szenario, bei dem die Nachfrage nach Kondensatoren aus Tantal und somit die Nachfrage durch ICT komplett verschwindet. Dies zeigt den Einfluss der Nachfrage nach Tantal durch ICT auf die Nachhaltigkeit auf.

\subsection{Zielsetzung}

Es werden die Dimensionen Wirtschaft, Gesellschaft und Umwelt für folgende Situationen untersucht:

\paragraph{2013}
Wie war die Situation im Jahr 2013?
\paragraph{2035}
Wie sieht die Situation bei gleichbleibend steigendem Tantalbedarf im Jahr 2035 aus?
\paragraph{2035 (Alternativszenario)}
Durch technolgische Innovation bricht der Bedarf nach Kondensatoren aus Tantal ein. Wie sieht die Situation bei sinkendem Tantalbedarf im Jahr 2035 aus?
