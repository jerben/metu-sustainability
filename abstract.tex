Die vorliegende Arbeit untersucht den Einfluss der ICT auf den Bedarf nach dem Metall Tantal. Das Ziel ist es, daduch die Nachhaltigkeit des Abbaus und Handels von Tantal zu bewerten.
Dazu werden die Situationen in 2013, 2035 und einem Alternativszenario im Jahr 2035 ohne Bedarf nach Tantal durch ICT erhoben. Für jedes Szenario untersucht man die Nachhaltigkeit in den Dimensionen Wirtschaft, Umwelt und Gesellschaft. Dabei zeigt man die wichtigsten drei Indikatoren pro Dimension auf und leitet eine Bewertung aus dem arithmetischen Mittel dieser Indikatoren her.
Bei der ganzheitlichen Analyse der Nachhaltigkeit von Tantal zeigt sich, dass der Abbau und Handel nicht nachhaltig sein kann. Nachhaltigkeit in einzeln betrachteten Dimensionen erachtet man hingegen für möglich. So zeigt sich durch die Gegenüberstellung der Szenarien, dass wirtschaftlich nachhaltige Förderung und Handel bei einem Einbruch der Nachfrage nach Tantal durch ICT möglich sind. Es zeigt sich zudem, dass die gesellschaftliche Nachhaltigkeit mit aktuellen Prognosen der Nachfrage im Jahr 2035 erreicht werden kann.
