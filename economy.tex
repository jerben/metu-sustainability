\section{Wirtschaftlich}\label{sec:conflict}

Dieser Abschnitt beschreibt die Nachhaltigkeit von Tantal aus wirtschaftlicher Sicht. Dabei bewertet man die aktuelle
Situation und die Prognosen nach ausgewählten Indikatoren. 

\subsection{Indikatoren}

Die Indikatoren für die Bewertung beschreibt man in diesem Abschnitt.

\subsubsection{Preis}

Mittels Indikator "Preis" untersucht man, wie sich der Preis pro Kilo Tantal in amerikanischen
Dollar (USD) verändert.

\subsubsection{Arbeitsplätze}

Der Indikator "Arbeitsplätze" misst, wie die Produktion von Tantal die Anzahl Arbeitsplätze beeinflusst.

\subsubsection{Investitionen}

"Investitionen" misst, ob das Investieren in Tantal gewinn- oder verlustbringend ist.

\subsubsection{Verursacherprinzip}

Der Indikator "Verursacherprinzip" zeigt auf, ob der durch die Produktion von Tantal entstandene Schaden
von der verursachenden Partei getragen wird.

\subsubsection{Innovation}

Dieser Indikator misst, ob durch die Produktion von Tental neue Technologien entstehen und ob diese
nachhaltig die Marktentwicklung beeinflussen können.

\subsection{Situation 2013}

\subsection{Situation 2035}

\subsubsection{Gleicher Tantalverbrauch}

\subsubsection{Alternative Technologie}

