%TODO: Die Schätzungen gehen nur von Veränderungen der verwendeten Menge bzgl.
%Transistoren aus. Wenn diese Menge nur ein Bruchteil ausmacht, dann hat auch
%die Verwendung einer Alternative kaum einen Einfluss. Das müssen wir noch
%irgendwie berücksichtigen.

\section{Wirtschaftlich}\label{sec:conflict}
Dieser Abschnitt beschreibt die Nachhaltigkeit von Tantal aus wirtschaftlicher
Sicht. Dabei bewertet man die aktuelle Situation und die Prognosen nach
ausgewählten Indikatoren.

\subsection{2013}
Fliesstext. Indikatoren im Lauftext hervorheben.


\subsection{2035 bei anhaltendem Trend}
Fliesstext. Indikatoren im Lauftext hervorheben.

\subsection{2035 mit Kondensatorenalternative}
Fliesstext. Indikatoren im Lauftext hervorheben.

\iffalse
\subsection{Indikatoren}
Die Indikatoren für die Bewertung beschreibt man in diesem Abschnitt.

\subsubsection{Preis}
Mittels Indikator ``Preis'' untersucht man, wie sich der Preis pro Kilo Tantal
in amerikanischen Dollars (USD) verändert.

\subsubsection{Investitionen}
``Investitionen'' misst, ob das Investieren in Tantal gewinn- oder
verlustbringend ist.

\subsubsection{Arbeitsplätze}
Der Indikator ``Arbeitsplätze'' misst, wie die Produktion von Tantal die Anzahl
Arbeitsplätze beeinflusst.

\subsubsection{Verursacherprinzip}
Der Indikator ``Verursacherprinzip'' zeigt auf, ob der durch die Produktion von
Tantal entstandene Schaden von der verursachenden Partei getragen wird.

\subsubsection{Innovation}
Dieser Indikator misst, ob durch die Produktion von Tental neue Technologien
entstehen und ob diese nachhaltig die Marktentwicklung beeinflussen können.

\subsection{Situation 2013}
In diesem Abschnitt geht man auf die aktuelle Situation der Tantalproduktion ein
und bewertet die zuvor erwähnten Indikatoren.

\paragraph{Preis}
Man schätzt den Preis für ein Kilo Tantal im Jahr 2018 auf USD
151.80.\cite{tantal_price} Dieser Preis ist als neutral bewertet.

\paragraph{Investitionen}
Investitionen sind eng mit dem Preis gekoppelt. Steigt der Preis, kann mit einer
Investition in Tantal ein Gewinn erzielt werden. Da die Bewertung für den
Indikator ``Preis'' neutral ausfällt, bewerten man diesen Indikator ebenfalls
neutral.

\paragraph{Arbeitsplätze}
%TODO: Quelle?

\paragraph{Verursacherprinzip}
%TODO: Sie aktuell nicht so gut aus... brauche aber noch eine Quelle

\paragraph{Innovation}
Die aktuelle Situation der Tantalproduktion zeigt keine Innovation. Dieser
Indikator ist neutral bewertet.

\subsection{Situation 2035}
In den folgenden zwei Abschnitten bewertet man die Indikatoren für zwei
Prognosen des Tantalkonsums.

\subsubsection{Steigender Tantalverbrauch}
Basierend auf der Prognose von TODO:LINK TO FIGURE untersucht man im folgenden
Abschnitt die Indikatoren bei steigendem Tentalverbrauch durch die Produktion
von Mikrotransistoren.

\paragraph{Preis}
%TODO: Link zu Figure, Quellen?
Bei einem grösseren Tantalverbrauch steigt die Nachfrage. Eine steigende
Nachfrage hat zur Folge, dass sich der Preis ebenfalls erhöht. Daraus folgt eine
negative Bewertung dieses Indikators.

\paragraph{Investitionen}
%TODO: Prognose als Quelle verwenden?
Durch einen steigenden Preis in Folge einer steigenden Nachfrage kann man
mittels einer Investition in Tantal einen Gewinn erzielen. Somit schätzt man den
Indikator ``Investitionen'' positiv ein.

\paragraph{Arbeitsplätze}

\paragraph{Verursacherprinzip}
%TODO: Durch neue Regulierungen der EU und der USA sehe ich in diesem Punkt eine
%Verbesserung, die Quelle fehlt einfach noch

\paragraph{Innovation}
Da nach wie vor Tantal für Mikrotransitoren verwendet wird, erkennt man keine
Veränderung der Innovation. Der Indikator bewertet man weiterhin als neutral.

\subsubsection{Alternative Technologie}
In diesem Abschnitt bewertet man die ausgewählten Indikatoren basierend auf der
Annahme, dass bis 2035 durch einen technologischen Fortschritt kein Tantal mehr
für Mikrotransistoren benötigt wird.

\paragraph{Preis}
%TODO: Link zu Figure, Quellen?
%TODO: Nur weil der Transistorteil wegfällt, heisst das nicht, dass der Preis
%sinken wird.
Wenn eine Alternative zu Tantal für die Herstellung von Transistoren gefunden
wird, sinkt die Nachfrage des Metalls. Der Preis wird voraussichtlich sinken,
was zu einer positiven Bewertung dieses Indikators führt.

\paragraph{Investitionen}
Bei einem sinkenden Preis verliert man bei Investitionen in Tantal Geld. Das
führt zu einer negativen Bewertung des Indikators.

\paragraph{Arbeitsplätze}

\paragraph{Verursacherprinzip}
Eine nachhaltige Alternative zu Tantal hat zur Folge, dass auch der Verursacher
alle Kosten für die Beschaffung des Materials tragen muss. Das ist eine
Verbesserung der aktuellen Situation und wird als positiv bewertet.

\paragraph{Innovation}
Eine neue Technologie, welche gegenüber Tantal nachhaltig ist, ist innovativ.
Man erreicht dasselbe ohne die negativen Nebeneffekte der Tantalproduktion. Der
Indikator ``Innovation'' erhält deswegen eine positive Bewertung.

\fi
