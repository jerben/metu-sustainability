\section{Gesellschaft}\label{sec:society}

\subsection{Einleitung}

\subsection{Indikatoren}

\subsubsection{Soziale Sicherheit}

Konflikte im Bezug auf Ressourcenabbau.
Finanzierung von Konfliktparteien.

\subsubsection{Gesundheit}

Gesundheit der Minenarbeiter.
Folgen von möglicher Umweltverschmutzung.
Statistiken zu Unfallraten in der Tantalproduktion.

\subsubsection{Solidaritaet}

Gegen die Ausbeutung von armen Laendern.
Bewusstsein der Konsumenten.
Offizielle Massnahmen  z.B. EU etc.

\subsubsection{Chancengleichheit}

Bergbau als Mittel gegen Armut.
Bei hoher Nachfrage höhere Preise -> Australien steigt wieder ein -> Arbeitsplaetze.

\subsection{Bewertung der Indikatoren}

Tabelle mit Bewertung der jeweiligen Indikatoren basierend auf unserer wissenschaftlichen Skala.


@misc{USGSMine8:online,
author = {Désirée E. Polyak},
title = {USGS Minerals Information: Niobium (Columbium) and Tantalum},
howpublished = {\url{https://minerals.usgs.gov/minerals/pubs/commodity/niobium/}},
month = {},
year = {},
note = {(Accessed on 09/13/2018)}
}

