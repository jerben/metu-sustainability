\section{Gesellschaft}\label{sec:society}

Der gesellschaftliche Einfluss auf die Nachhaltigkeit des Anstiegs der
prognostizierten Tantalproduktion ist global Verteilt. Im Rahmen dieser Analyse
wird der Schwerpunkt auf die wichtigsten Herkunftslaender für die
Tantalproduktion relevanten Mineralien gelegt.

% TODO Figure?

\subsection{Indikatoren}

Nachfolgend werden jeweils die einzelnen Indikatoren beschrieben, welche die
Basis für die gesellschaftliche Entwicklung der Nachhaltigkeit bilden.

\paragraph{Soziale Sicherheit}

Unter der sozialen Sicherheit werden Faktoren im Zusammenhang mit der
politischen Situation der Herkunftslaender zusammengefasst. Insbesondere der
Beitrag an die innenpolitische Stabilitaet des Landes traegt massgeblich zur
gesellschaftlichen Nachhaltigkeit bei.

\paragraph{Gesundheit}

Die Arbeitsverhaeltnisse in den Produktionsstaetten für Tantal und die Folgen
der Umweltverschmutzung resultierend aus der Produktion auf die lokale
Bevölkerung sind die wichtigsten Aspekte für den Indikator Gesundheit.

\paragraph{Solidaritaet}

Der Indikator zur Solidaritaet umfasst das Bewusstsein der Konsumenten und
Maerkte zur Herkunft und Produktion von Tantal.

\paragraph{Chancengleichheit}

Die Chancengleichheit umfasst das Verhaeltniss von Arbeitnehmer und
Arbeitnehmer in den Produktionsstaetten, sowie den Einfluss auf die lokale
Bevölkerung.

\subsection{Bewertung}

Gemaess dem U.S. Geological Survey stammen über 60\% der Mineralien für die
Tantalproduktion aus Zentralafrika ~\cite{USGSMine8}. Die H\"alfte davon
wird in der Demokratischen Republik Kongo abgebaut, welches als eines der \"armsten
L\"andern der Welt gilt gemessen am Human Development Index ~\cite{UNDProgramme2018}. Der Abbau von Mineralien im
Kongo wird von den verschiedenen Konfliktparteien kontrolliert, welche wiederum
den Erlös aus den Minen zur Finanzierung des Konflikts nutzen. Aus diesem Grund
wird Tantal als Konfliktmineral klassifiziert ~\cite{doevenspeck2012konfliktmineralien}.
Gesellschaftlich führt dies zu einem negativen Einfluss im Bereich der sozialen
Sicherheit, da ein wesentlicher Teil des Gesamtvolumens aus Konfliktregionen
stammt und so zur destabilisierung der betroffenen Regionen beitr\"agt. [Insert citation]
Im Bezug auf die Gesundheit wirkt sich die Tantalgewinnung ebenfalls negativ auf, da die 
Produktionss\"atten in den Konlfiktgebieten ohne Rücksicht auf die Arbeiter und umliegende Umwelt
operieren. [insert citation]

Im Bezug auf den Indikator zur Solidaritaet kann eine positive Entwicklung vermerkt werden.
Dies zeigt sich an den zunehmenden Bemühungen die Einfuhr von Konfliktmineralien 
aufzudecken und Einhalt zu gebieten. Als Beispiel sagt die EU-Verordnung zu Konfliktmineralien[insert citation], welche im Jahr 2017 in Kraft getreten ist, dass alle Tantal-Importe in die EU die Standards zur nachhaltigen Beschaffung, definiert durch die OECD, erfüllen müssen 
~\cite{europeancommission}. Die Effektivit\"at solcher Massnahmen muss sich aber erst zeigen. 
So gibt es bereits eine Studie von Germanwatch zur EU-Verordnung, welche kritisiert, dass 
"Unternehmen mit Scheinlösungen davonkommen könnten" ~\cite{Governan35}.

In den Konfliktgebieten des Kongo und den Nachbarslaendern wie Ruanda ist der Abbau von 
Mineralien zur Produktion von Tantal eine wichtige Einkommensquelle für die lokale Bevölkerung
und eines der Haupt-Exportgüter der jeweiligen Laender [insert citation]. Dies fliesst in
die Bewertung der Chancengleichheit ein, da ohne den Export dieser Mineralien die Verarmung
der Bevölkerung stark ansteigen würde ~\cite{DRCongo35}.


\begin{table}[]
    \begin{tabular}{l|lll} & \textbf{2013} & \textbf{2035} &  \textbf{2035 mit Annahme} 
        \\ \hline Soziale Sicherheit    & 2  & 5  & 5
        \\ Gesundheit                   & 3  & 5  & 5
        \\ Solidarität                  & 6  & 5  & 5
        \\ Chancengleichheit            & 6  & 5  & 3
        \\ \hline \textbf{Endbewertung} & 4  & 5  & 5
    \end{tabular}
\end{table}
